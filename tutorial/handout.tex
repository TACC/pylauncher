\documentclass[12pt]{artikel3}

\usepackage{pslatex}

\def\n#{\bgroup \catcode`\_=12 \catcode`\>=12 \catcode`\<=12
  \catcode`\&=12 \catcode`\^=12 \catcode`\~=12 \def\\{\char`\\}\relax
  \tt \let\next=}

\title{Pylauncher Lab}
\author{Victor Eijkhout}
\date{Throughput Computing Workshop, September 27}
\begin{document}
\maketitle

\section{Setting up the environment for the tutorial}

\begin{itemize}
\item Make sure you have python available:
\begin{verbatim}
module load python
\end{verbatim}
\item Make sure the pylauncher is available:
  \begin{itemize}
  \item Normally you would do:
\begin{verbatim}
module load pylauncher
\end{verbatim}
  \item but for this lab we'll use the latest version of the launcher
\begin{verbatim}
tar fxz ~train00/pylauncher.tar.gz
cd pylauncher
\end{verbatim}
  \end{itemize}
\item You can run the examples by submitting batch jobs, but here we will do everything interactively.
Request one compute node with
\begin{verbatim}
/usr/bin/srun -n 16 -p normal \
  --reservation="TACC-Training-2013-09-27" -A PROJECTNUMBER \
  -t 0:30:00 --pty /bin/bash -l
\end{verbatim}
(for two nodes, change the 16 to 32, et cetera)
where \n{PROJECTNUMBER} is
\begin{itemize}
\item  \n{20130927DataIntensive} if you signed up through TACC, or
\item \n{TG-TRA120009} if you signed up through XSEDE.
\end{itemize}
\item You should now be in the \n{pylauncher} directory. Do:
\begin{verbatim}
export PYTHONPATH=`pwd`:$PYTHONPATH
\end{verbatim}
(this is normally not necessary if you load the module)
and go to the tutorial directory:
\begin{verbatim}
cd tutorial
\end{verbatim}
\end{itemize}

\section{A simple launcher example}

\begin{itemize}
\item Run the python file \n{random_commandline.py} to generate a file with commandlines. Inspect the resulting file.
\item Compile the \n{random_wait} program:
\begin{verbatim}
cc -O -o random_wait random_wait.c
\end{verbatim}
\item Run the example:
\begin{verbatim}
python run_random.py
\end{verbatim}
after a minute or two you get summary output.
\item You should now have a directory with a name
  \n{pylauncher_tmpdir123456}. The number is your job number. An
  interactive session with \n{srun} is actually a type of batch job,
  so you get the number of that job.
  \begin{itemize}
  \item Take a look in that directory. Do you see that for each task number there are three files?
  \item  Do \n{ls -s}. You'll see that files with a name \n{expire123} have zero size. 
  \item Take a look at one of the \n{exec123} files. On their last
    line is the command that they should execute, followed by a
    \n{touch} command that is the source of the \n{expire00} file.
  \item Finally, there are the \n{out99} files, which contain the
    standard output (and stderr) of the corresponding \n{exec} file.
  \end{itemize}
\item Since we're running interactively, maybe you want some output on
  what the launcher is doing. Add \n{debug="job"} as parameter to the
  launcher call.
\item Try running the example again. Take a close look at the error
  message.\par The PyLauncher wants a unique working directory for
  each run, and since interactively we are still in the same run you
  need to remove the working directory of the previous python call
  with \n{rm -rf pylauncher_tmpdir1234567}. Now you can run the example again.
\item The output scrolls by rather fast. Use Control-S to stop it and
  Control-Q to resume. When it's stopped, can you understand what it's
  reporting?
\end{itemize}

\section{Further examples}

\begin{itemize}
\item If you use \n{PYL_ID} in a commandline, it gets replaced dynamically by the task number.
\item Edit the \n{random_commandlines.py} file to use \n{PYL_ID} instead of a random number.
\item Run the \n{run_random.py}.
\item Go into the workdirectory that was generated. Take a look at a
  few \n{out99} files. Can you confirm that everything worked the way
  it should?
\end{itemize}

\end{document}
